\documentclass{article}
\usepackage[banglamainfont=Kalpurush, 
            banglattfont=Kalpurush
           ]{latexbangla}
\usepackage[top=8em, left=10em, right=10em, bottom=8em]{geometry}
\begin{document}


{\Large সোনার বাংলা}\

এটি একটি ইংরেজি লেখা ছোট অধ্যায় যে বাঙ্গালী এ দ্বারা অনুবাদ করা হয়েছে. এটা খুব স্পষ্ট নয় যদি সঠিক অনুবাদ বা না কিন্তু ক্রিয়াটি ফন্ট দেখাতে যথেষ্ট হওয়া উচিত.
পিথাগোরাস(Pythagoras)-এর উপপাদ্যটি হল,\\
\textit{সমকোণী ত্রিভুজের অতিভুজের উপর অঙ্কিত বর্গক্ষেত্রের ক্ষেত্রফল অপর দুই বাহুর 
উপর অঙ্কিত বর্গক্ষেত্রের ক্ষেত্রফলের সমষ্টির সমান।}\\
অর্থাৎ কোন সমকোণী ত্রিভুজের অতিভুজ $c$ এবং অপর দুই বাহু $a$ এবং $b$ হলে,
\[c^2=a^2+b^2\]
লক্ষ্য করুন, এখন পর্যন্ত টেক্সট প্রদর্শনের জন্য \textbf{কালপুরুষ} ফন্ট ব্যবহৃত হয়েছে।


\begin{problem}
$p$ একটি মৌলিক সংখ্যা এবং $n$ একটি স্বাভাবিক সংখ্যা হলে প্রমাণ কর যে 
$p|n^p-n$ 
\end{problem}
\begin{proof}
$p|n$ হলে এটি স্বাভাবিকভাবেই সত্যি। অন্যথায়, $\{n,2n,\ldots,(p-1)n\}$
 একটি পূর্ণাঙ্গ রেসিডিও ক্লাস তৈরি করে। যার ফলে...
\end{proof}
\begin{problem}
$a$ এবং $n$ পরস্পর সহমৌলিক স্বাভাবিক সংখ্যা হলে দেখাও যে $a^{\phi(n)}
\equiv 1\pmod n$
\end{problem}

\begin{problem}
$p$ একটি মৌলিক সংখ্যা এবং $n$ একটি স্বাভাবিক সংখ্যা হলে প্রমাণ কর যে $p|n^p-n$ 
\end{problem}
\begin{proof}
$p|n$ হলে এটি স্বাভাবিকভাবেই সত্যি। অন্যথায়, $\{n,2n,\ldots,(p-1)n\}$ একটি
পূর্ণাঙ্গ রেসিডিও ক্লাস তৈরি করে। যার ফলে...
\end{proof}
\begin{problem}
$a$ এবং $n$ পরস্পর সহমৌলিক স্বাভাবিক সংখ্যা হলে দেখাও যে $a^{\phi(n)}\equiv 1\pmod n$
\end{problem}

\begin{table}
  \centering
  \caption{Example of Statics Tabular format}

  \begin{tabular}{l  c  c   c  c  p{5.cm}}
  \hline\hline
   Parameter & P+WP & P+WP+highL & P+WP+highL+L & P+WP+highL+L+BAO & Summary \\ \hline
    $\Omega_bh^2$ & 0.022032 & 0.022069 & 0.022199 & 0.022161 & Baryon    \\ 
    $\Omega_ch^2$& 0.12038 & 0.12025 & 0.11847 & 0.11889 & Cold dark matter \\
    $\Omega_mh^2$& 0.14305 & ... & ... & ... & Total matter \\ 
    $\Omega_{\Lambda}$ & 0.6817 & 0.6830 & 0.6939 & 0.6914 & Dark energy \\
    $\Sigma m_{\nu}$[eV]&0.002&...&0.000&...&neutrino\\
    $H_0$& 67.04&67.15&67.94&67.77&expansion rate \\\hline
    
  \end{tabular}
  
  \caption{Example of Dynamic abular format}
  \begin{tabular}{l  c  c   c  c  p{5.cm}}
  \hline
    Date & \multicolumn{1}{c} {Flow} \tabularnewline \hline    
    %{ for row in data%}
      %{{ row.date %}} & %{{ row.flow %}} \tabularnewline \hline
    %{ endfor %}
   
  \end{tabular}
\end{table}

\end{document}
