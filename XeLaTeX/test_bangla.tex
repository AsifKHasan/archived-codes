\documentclass[10pt]{article}
\usepackage{fontspec}
\usepackage{polyglossia}
\setdefaultlanguage[numerals=Devanagari]{bengali}
\setotherlanguage{english}
\setmainfont[Script=Bengali]{Lohit Bengali}

%\newcommand{\devanagarinumeral}[1]{%
%  \devanagaridigits{\number\csname c@#1\endcsname}}

% renew all representation of counters
% \renewcommand{\thesection}{\devanagarinumeral{section}}
% \renewcommand{\thepage}{\devanagarinumeral{page}}
% \renewcommand{\theenumi}{\devanagarinumeral{enumi}}

\title{অতঃপর বাংলা লিখতে পেরেছি :) টেক্স মেকার এডিটর থেকে }
\date{\today}

\begin{document}
\maketitle

\section{যুক্ত বর্ণ} যেমন- অক্ট্রয় ক্ত = ক ত;
                    যেমন- রক্ত ক্ত্র = ক ত র;
                    যেমন- বক্ত্র ক্ব = ক ব;
                    যেমন- পক্ব, ক্বণ ক্ম = ক ম;
                    যেমন- রুক্মিণী ক্য = ক য;
                    যেমন- বাক্য ক্র = ক র;
                    যেমন- চক্র ক্ল = ক ল;
                    যেমন- ক্লান্তি ক্ষ = ক ষ;
                    যেমন- পক্ষ ক্ষ্ণ = ক ষ ণ;
                    যেমন- তীক্ষ্ণ ক্ষ্ব = ক ষ ব;
                    যেমন- ইক্ষ্বাকু ক্ষ্ম = ক ষ ম;
                    যেমন- লক্ষ্মী ক্ষ্ম্য = ক ষ ম য;
                    যেমন- সৌক্ষ্ম্য ক্ষ্য = ক ষ য;
                    যেমন- লক্ষ্য ক্স = ক স;
                    যেমন- বাক্স খ্য = খ য;
                    যেমন- সখ্য খ্র = খ র যেমন;
                    যেমন- খ্রিস্টান গ্ণ = গ ণ; যেমন – রুগ্ণ গ্ধ = গ ধ;
                    যেমন- মুগ্ধ গ্ধ্য = গ ধ য;
                    যেমন- বৈদগ্ধ্য গ্ধ্র = গ ধ র;
                    যেমন- দোগ্ধ্রী গ্ন = গ ন;
                    যেমন- ভগ্ন গ্ন্য = গ ন য;
                    যেমন- অগ্ন্যাস্ত্র, অগ্ন্যুৎপাত, অগ্ন্যাশয় গ্ব = গ ব;
                    যেমন- দিগ্বিজয়ী গ্ম = গ ম;
                    যেমন- যুগ্ম গ্য = গ য;
                    যেমন- ভাগ্য গ্র = গ র;
                    যেমন- গ্রাম গ্র্য = গ র য;
                    যেমন- ঐকাগ্র্য, সামগ্র্য, গ্র্যাজুয়েট গ্ল = গ ল;
                    যেমন- গ্লানি ঘ্ন = ঘ ন;
                    যেমন- কৃতঘ্ন ঘ্য = ঘ য;
                    যেমন- অশ্লাঘ্য ঘ্র = ঘ র;
                    যেমন- ঘ্রাণ ঙ্ক = ঙ ক;
                    যেমন- অঙ্ক ঙ্ক্ত = ঙ ক ত;
                    যেমন- পঙ্ক্তি ঙ্ক্য = ঙ ক য;
                    যেমন- অঙ্ক্য ঙ্ক্ষ = ঙ ক ষ;
                    যেমন- আকাঙ্ক্ষা ঙ্খ = ঙ খ;
                    যেমন- শঙ্খ ঙ্গ = ঙ গ;
                    যেমন- অঙ্গ ঙ্গ্য = ঙ গ য;
                    যেমন- ব্যঙ্গ্যার্থ, ব্যঙ্গ্যোক্তি ঙ্ঘ = ঙ ঘ;
                    যেমন- সঙ্ঘ ঙ্ঘ্য = ঙ ঘ য;
                    যেমন- দুর্লঙ্ঘ্য ঙ্ঘ্র = ঙ ঘ র;
                    যেমন- অঙ্ঘ্রি ঙ্ম = ঙ ম;
                    যেমন- বাঙ্ময় চ্চ = চ চ;
                    যেমন- বাচ্চা চ্ছ = চ ছ;
                    যেমন- ইচ্ছা চ্ছ্ব = চ ছ ব;
                    যেমন- জলোচ্ছ্বাস চ্ছ্র = চ ছ র;
                    যেমন- উচ্ছ্রায় চ্ঞ = চ ঞ;
                    যেমন- যাচ্ঞা চ্ব = চ ব;
                    যেমন- চ্বী চ্য = চ য;
                    যেমন- প্রাচ্য জ্জ = জ জ;
                    যেমন- বিপজ্জনক জ্জ্ব = জ জ ব;
                    যেমন- উজ্জ্বল জ্ঝ = জ ঝ;
                    যেমন- কুজ্ঝটিকা জ্ঞ = জ ঞ;
                    যেমন- জ্ঞান জ্ব = জ ব;
                    যেমন- জ্বর জ্য = জ য;
                    যেমন- রাজ্য জ্র = জ র;
                    যেমন- বজ্র ঞ্চ = ঞ চ;
                    যেমন- অঞ্চল ঞ্ছ = ঞ ছ;
                    যেমন- লাঞ্ছনা ঞ্জ = ঞ জ;
                    যেমন- কুঞ্জ ঞ্ঝ = ঞ ঝ;
                    যেমন- ঝঞ্ঝা ট্ট = ট ট;
                    যেমন- চট্টগ্রাম ট্ব = ট ব;
                    যেমন- খট্বা ট্ম = ট ম;
                    যেমন- কুট্মল ট্য = ট য;
                    যেমন- নাট্য ট্র = ট র;
                    যেমন- ট্রেন (মন্তব্য: এই যুক্তাক্ষরটি মূলত ইংরেজী/ বিদেশী কৃতঋণ শব্দে ব্যবহৃত) ড্ড = ড ড;
                    যেমন- আড্ডা ড্ব = ড ব;
                    যেমন- অন্ড্বান ড্য = ড য;
                    যেমন- জাড্য ড্র = ড র;
                    যেমন- ড্রাইভার, ড্রাম (মন্তব্য: এই যুক্তাক্ষরটি মূলত ইংরেজী/ বিদেশী কৃতঋণ শব্দে ব্যবহৃত) ড়্গ = ড় গ;
                    যেমন- খড়্গ ঢ্য = ঢ য;
                    যেমন- ধনাঢ্য ঢ্র = ঢ র;
                    যেমন- মেঢ্র (ত্বক) (মন্তব্য: অত্যন্ত বিরল) ণ্ট = ণ ট;
                    যেমন- ঘণ্টা ণ্ঠ = ণ ঠ;
                    যেমন- কণ্ঠ ণ্ঠ্য = ণ ঠ য;
                    যেমন- কণ্ঠ্য ণ্ড = ণ ড;
                    যেমন- গণ্ডগোল ণ্ড্য = ণ ড য;
                    যেমন- পাণ্ড্য ণ্ড্র = ণ ড র;
                    যেমন- পুণ্ড্র ণ্ঢ = ণ ঢ;
                    যেমন- ষণ্ঢ ণ্ণ = ণ ণ;
                    যেমন- বিষণ্ণ ণ্ব = ণ ব;
                    যেমন- স্হাণ্বীশ্বর ণ্ম = ণ ম;
                    যেমন- চিণ্ময় ণ্য = ণ য;
                    যেমন- পূণ্য ৎক = ত ক;
                    যেমন- উৎকট ত্ত = ত ত;
                    যেমন- উত্তর ত্ত্ব = ত ত ব;
                    যেমন- সত্ত্ব ত্ত্য = ত ত য;
                    যেমন- উত্ত্যক্ত ত্থ = ত থ;
                    যেমন- অশ্বত্থ ত্ন = ত ন;
                    যেমন- যত্ন ত্ব = ত ব;
                    যেমন- রাজত্ব ত্ম = ত ম;
                    যেমন- আত্মা ত্ম্য = ত ম য;
                    যেমন- দৌরাত্ম্য ত্য = ত য;
                    যেমন- সত্য ত্র = ত র
                    যেমন- ত্রিশ, ত্রাণ ত্র্য = ত র য;
                    যেমন- বৈচিত্র্য ৎল = ত ল;
                    যেমন- কাৎলা ৎস = ত স;
                    যেমন- বৎসর, উৎসব থ্ব = থ ব;
                    যেমন- পৃথ্বী থ্য = থ য;
                    যেমন- পথ্য থ্র = থ র;
                    যেমন- থ্রি (three) (মন্তব্য: এই যুক্তাক্ষরটি মূলত ইংরেজী/ বিদেশী কৃতঋণ শব্দে ব্যবহৃত) দ্গ = দ গ;
                    যেমন- উদ্গম দ্ঘ = দ ঘ;
                    যেমন- উদ্ঘাটন দ্দ = দ দ;
                    যেমন- উদ্দেশ্য দ্দ্ব = দ দ ব;
                    যেমন- তদ্দ্বারা দ্ধ = দ ধ;
                    যেমন- রুদ্ধ দ্ব = দ ব;
                    যেমন- বিদ্বান দ্ভ = দ ভ;
                    যেমন- অদ্ভুত দ্ভ্র = দ ভ র;
                    যেমন- উদ্ভ্রান
\end{document}

% compile with xelatex: xelatex bangla.tex